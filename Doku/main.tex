\documentclass[12pt,a4paper,fleqn,leqno,openany]{article}

% Package Imports --------------------------------------------------------------

\usepackage{tikz}
\usepackage{tcolorbox}
\usepackage{graphicx}
\usepackage{listings}
\usepackage{xcolor}
\usepackage{lastpage}
\usepackage{fancyhdr}
\usepackage{hyperref}
\usepackage{float}

% Parameter Setup --------------------------------------------------------------

% Define Constant "true"
\newcommand{\true}{true}

% Define automatically filled in Information
\newcommand{\Schueler}{David Rieser}
\newcommand{\Lehrer}{Prof. Lezuo}
\newcommand{\Uebungsname}{Zephyr Serieller Krypto-Prozessor} % Name von der Uebung
\newcommand{\ToC}{true} % Table Of Contents - Inhaltsverzeichnis
\newcommand{\LoF}{true} % List Of Figures - Abbildverzeichnis
\newcommand{\LoL}{true} % List Of Listings - Codeverzeichnis

% Title, Header and Footer Setup -----------------------------------------------
\pagestyle{fancy}
\fancyhf{}

% Define Title-Page
\title{\Uebungsname}
\author{\Schueler\\Betreuer~:~\Lehrer}
% \date{} % When left out fills in current Date automatically

% Define Header and Footer Text
\rhead{5BHEL}
\lhead{\Uebungsname}
\rfoot{Page~\thepage ~of~\pageref{LastPage}}
\lfoot{HTL-Anichstrasse}

% Add Lines under the Header and over the Footer
\renewcommand{\headrulewidth}{1pt}
\renewcommand{\footrulewidth}{1pt}

% Graphics Setup ---------------------------------------------------------------

\graphicspath{{Pictures/}}

% Example Picture
%
% \begin{figure}[bht!]
% 	\centering
% 	\includegraphics[width=0.8\textwidth]{Bild.jpg}
% 	\caption{Bildbeschreibung}
%	\label{fig:Bildamrker}
% \end{figure}
%
% Link to Picture
%
% \ref{Referenz:Bild}

% Listings Setup ---------------------------------------------------------------

\lstset{
	alsoletter=-,
    basicstyle=\footnotesize\ttfamily,
    breaklines=true,
	captionpos=b,
	columns=fullflexible,
    frame=trBL,
	language=C,
	numbers=left,
	numbersep=10pt,
	numberstyle=\tiny,
	showstringspaces=false,
	showtabs=true,
	stepnumber=1,
	tab=,
	tabsize=4,
	title=\lstname,
	% Define Comments
    commentstyle=\color{lightgray},
	morecomment=[l][\color{lightgray}]{\\\\},
	morecomment=[l][\color{lightgray}]{//},
	morecomment=[l][\color{lightgray}]{\%},
	morecomment=[s][\color{lightgray}]{/*}{*/},
	morecomment=[s][\color{lightgray}]{\\*}{*\\},
	% Define Strings
	stringstyle=\color{orange},
	morestring=[s][\color{orange}]{\$}{\$}
}

% Example Listing
%
% \begin{lstlisting}[Options]
%	Code
% \end{lstlisting}
%
% Example File Import
%
% \lstinputlisting[Options]{filename}
%
% Define New Language
%
% \lstdefinelanguage{JavaScript}{
%   keywords={typeof, new, true, false, catch, function, return, null, catch, switch, var, if, in, while, do, else, case, break},
%   keywordstyle=\color{blue}\bfseries,
%   ndkeywords={class, export, boolean, throw, implements, import, this},
%   ndkeywordstyle=\color{darkgray}\bfseries,
%   identifierstyle=\color{black},
%   sensitive=false,
%   comment=[l]{//},
%   morecomment=[s]{/*}{*/},
%   commentstyle=\color{purple}\ttfamily,
%   stringstyle=\color{red}\ttfamily,
%   morestring=[b]',
%   morestring=[b]"
% }
%
% Define New Style
%
% \lstdefinestyle{StyleMatlab}
% {
%   belowcaptionskip=1\baselineskip,
%   breaklines=true,
%   xleftmargin=\parindent,
%   language=Matlab,
%   frame = single,
%   showstringspaces=false,
%   basicstyle=\footnotesize\ttfamily,
%   keywordstyle=\bfseries\color{black},
%   commentstyle=\color{green},
%   identifierstyle=\color{blue},
%   stringstyle=\color{magenta},
% }

% Hyperlink Setup --------------------------------------------------------------

\urlstyle{same}
\hypersetup{
    colorlinks=true,	% Enable Link-Colors
    linkcolor=blue, 	% Color of Internal-Document-Links
    filecolor=magenta, 	% Color of Local-File-Links
    urlcolor=cyan, 		% Color of URL-Links
}

% Hyperref Setup
%
% \label{Marker} 			% To setup the Marker
%
% \url{url}					% Clickable URL
% \href{url}{Link-Text}		% Clickable Text leading to URL
% \ref{Marker}				% To reference the Marker (Returns Number)
% \pageref{Marker}			% To reference the Marker (Returns Number)
% \nameref{Marker}			% To reference the Marker (Returns Name)

% ------------------------------------------------------------------------------
% Document Start ---------------------------------------------------------------
% ------------------------------------------------------------------------------

\begin{document}

% Disable Pagenumbering for Titlepage and Table of Contents
\pagenumbering{gobble}

% Create Title-Page
\begin{titlepage}
	\maketitle
\end{titlepage}

\pagebreak

% Remove Header and Footer for Table of Contents
\thispagestyle{empty}

\ifx\ToC\true
	\renewcommand\contentsname{Inhaltsverzeichnis}
	\tableofcontents
\fi
\pagebreak
\ifx\LoF\true
	\let\oldnumberline\numberline
	\renewcommand{\numberline}[1]{\hspace*{-1.5em}}
	\renewcommand\listfigurename{Abbildungsverzeichnis}
	\listoffigures
	\let\numberline\oldnumberline
\fi
\ifx\LoL\true
	\renewcommand\lstlistlistingname{Codeverzeichnis}
	\lstlistoflistings
\fi

\pagebreak

\pagenumbering{arabic}

% aus der Angabe kopiert:
%
% Dokumentation
% Dokumentieren und Beschreiben Sie: west, ninja, Kconfig,
% DTS, threads und message_queues (ca. 8 Seiten)
%
% zu allen genannten Punkten:
%
% * welches Problem löst es?
% * wie ist es umgesetzt?
% * wie unterscheidet es sich alternativen Lösungen?
%
% Der Code dokumentiert sich hoffentlich selbst, bzw. hat an kritischen Stellen
% Source Kommentare,
% d.h. nicht Teil der Dokumentation.
%
% Du solltest dich auf ca. 8 Seiten beschränken, d.h. auf die wichtigsten Dinge
% (was auch immer du für wichtig hältst und diese Auswahl ist Teil der Bewertung ;)

\section{Used Technologies}

\subsection{Zephyr}

Zephyr is a small real-time operating system for embedded devices with constraints
on hardware resources which supports multiple different architectures.\\
Zephyr is released under the Apache License 2.0.

\subsubsection{West}

\subsubsection{Ninja}

\subsubsection{CMake}


\subsection{Linux}

\subsubsection{Virtual Terminals}

\pagebreak

\section{Project Description}

The Goal of the Project is to create a Processor which can receive data, then
encrypt or decrypt this data using AES-128 and then send it back via a
serial-Interface.

% \input{build/Section2.tex}
% \input{build/Section3.tex}
% \input{build/Section4.tex}

\end{document}
